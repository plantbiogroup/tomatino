\documentclass{article} \usepackage{multicol}
\setlength{\columnsep}{20pt}
\begin{document}

\section{The chassis}
The task of the \textit{chassis} is to hold the components in place.
It also gives a place to organize the different pieces of your
project.

\begin{multicols}{2}

  \subsection{Materials}
  \begin{itemize}
  \item A sheet of Plexiglass at least 300x50 cm about 6mm thick.
  \item 8 standoffs 4-40
  \item 8 sunken flath head 4-40 screws to fasten the standoffs on to
    the Plexiglass.
  \item 8 screws to fasten the PC boards to the standoffs.
  \item 16 non conductive washers. Nylon washers are good.
  \end{itemize}

  \subsection{Tools}
  \begin{itemize}
  \item A drill, \#30, or \#32, or 3/32 inch will do.
  \item A counter sink
  \end{itemize}

  \subsection{Procedure}
  \begin{enumerate}
  \item Mark the center of the holes of the PC boards on the
    Plexiglass base with a punch

  \item Drill the holes.

  \item Countersink the holes on the bottom.  This gives you a nice
    flush bottom surface.
  \end{enumerate}

  \subsection{Notes}
  Depending on the type of boards you get, your layout will vary.  If
  you make a mistake, Plexiglass is quite forgiving, and you can drill
  until you get it right.

  A nice substitute for standoffs is an ordinary screw with nuts on
  both sides of the PC board.  You can also make hillbilly standoffs
  by cutting straws or a vinyl hose into bits and use them as
  standoffs.
\end{multicols}

\section{The Relay Adaptor Board}
The \textit{Realy Adaptor Board} is holds the filters for the relay.
The relay bank can be attached directly to your Arduino, but chances
are that the electric noise of the relay will feed back to the Arduino
and mess with it.

In the firs iteration of the \textit{TomatoPi} we hooked the relay
bank directly up to the GPIO pins of a Raspberry Pi.  This setup
worked initially, but after a while we got intermittent crashes on the
Pi.

A Relay Adaptor Board can save you many debugging headaches.

The circuit is a simple pull-down resistor with a bypass capacitor in
parallel, repeated once for each input signal.  A larger capacitor is
hooked up from Vcc to ground in order to smooth any spikes, and
provide a bit of extra juice to the relays.

\begin{multicols}{2}
  \subsection{Tools}
  \begin{itemize}
  \item Soldering iron.  Buy yourself a nice one with hand protection.
  \item Solder.
  \item Small wire cutter
  \item Volt meter for testing the circuit.
  \end{itemize}

  \subsection{Materials}
  \begin{itemize}
  \item 8 x 1k resistors.  I used 1/16W 5%
  \item 8 x 10 nF Ceramic capacitors.  Most any small capacitor will
    do.
  \item 1 x 100 uF Electroltic capacitor. Rated at least two times
    higher than the feeding voltage.  I feed with 5v so I use a 10v
    capacitor.  It is a good rule of thumb.
  \item 2 x 10 pin stackable header.  You can use other headers, but
    the stackable ones are handy for prototyping.
  \item 1 x Double sided 70x50mm pre drilled proto-board (or larger)
  \item 22 gauge hookup wire. Different colors for Vcc, signal and
    ground.
  \end{itemize}

  \subsection{Procedure}
  Lay out the board and solder it.... ;)

  Test the connections with your volt meter.
\end{multicols}


\section{Realy Housing and Outlets}
Note. You are explicitly forbidden to hook this up to the 110v power
main unless you have proper authorization.  If you do that,
\begin{multicols}{2}
  \subsection{Foo}
  bar baz.
\end{multicols}
\end{document}
